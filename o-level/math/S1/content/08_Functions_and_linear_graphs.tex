\documentclass[../main]{subfiles}

\begin{document}

\section{Functions and Linear Graphs}

\subsection{Cartesian Coordinates}

The \textbf{Cartesian coordinate system} describes the positions of points and
lines in a plane. A Cartesian plane consists of a horizontal axis and a vertical
axis.
\begin{enumerate}
\item The horizontal axis: \textbf{x-axis}
\item  The vertical axis: \textbf{y-axis}
\item The two axes intersect at the origin, \(O\).

\end{enumerate}

The position of any point on the Cartesian plane is represented as an ordered
pair \((a,b)\) of real numbers know as \textbf{coordinates}.

Note:
\(a\) is know as the \(x-coordinate\)
\(b\) is know as the \(y-coordinate\)

\begin{tikzpicture}[scale=1.5]
  \coordinate (0) at (0,0);
  \draw [->] (-2, 0 ) -- (2,0);
  \draw [<-] (0,2) -- (0,-2) ;

\end{tikzpicture}
\end{document}