\documentclass[../main]{subfiles}

\pgfplotsset{every axis/.append style={
                    axis x line=middle,    % put the x axis in the middle
                    axis y line=middle,    % put the y axis in the middle
                    axis line style={<->}, % arrows on the axis
                    xlabel={$Re(z)$},          % default put x on x-axis
                    ylabel={$Im(z)$},          % default put y on y-axis
                    }}
\tikzset{>=stealth}

\begin{document}

\section{Basic Algebra}

\subsection{Fundamental Algebra}
In algebra, letters are used to represent numbers.

A \textbf{variable} is a letter that is used to represent an unknown number or
some unknown number(s).

An \textbf{algebraic expression} involves numbers and variables that are
connected with operational signs(such as \(+, -, \times and \div\)).

\(5x\) is \textbf{not an algebraic expression}, \(5x\) is a term.

\(8p - 2q\) is \textbf{an algebraic expression}.

\(2a +7b = 105\) is an algebraic equation.

The terms in algebra can be expressed in \textbf{index notation}. For example,
\(p \times p \times p\) can be expressed as \(p^3\).

Common mistake:
\(36m + 6n = 42mn \) is wrong! Terms of different variables cannot be added.
\(12m \times 3 + 54mn \div 9m = \frac {36m + 54mn} 9m = 4 + 6n\) is wrong!
Always perform multiplication and division before addition and subtraction.

\subsection{Evaluation of algebraic expressions and formulae}
To evaluate an algebraic expression, substitute each variable with the given
value.

A \textbf{formula} is general mathematical statement or rule.
Given that p = -10, q =2, and r = 6, evaluate 
\(8p - 19q +25r\)
\begin{align*}
 8p-19q + 25 r &= 8(-10) - 19(2)+ 25(6)\\
               &= -80-38 +150 \\
               &=\mathbf{32}
\end{align*}

\section{Algebraic Manipulation}
\subsection{Like Terms and Unlike Terms}

\end{document}