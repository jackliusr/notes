\documentclass[../main]{subfiles}

\begin{document}

\section{Basic Geometry}
\subsubsection{Points, Lines and Planes}
The most basic geometric figure is a \textbf{point}. All other geometric figures
are made up of a collection of points. A point has a position but it has neither size nor shape. 

Lines:

When we join two points A and B together, a straight \textbf{line segment} AB is
formed. A and B are called the \textbf{endpoints}.

If we extended the line segment AB in both directions indefinitely, we get a
line. 

ray:

point of intersection

Plane:
A \textbf{plane} is a flat surface in which any two points are joined by a line
that lies entirely on the surface.  horizontal plane , vertical plane and curved
surface.

\subsubsection{Angles}

When two rays $OA$ and $OB$ share a common point $O$, an \textbf{angle} is formed.
$O$ is known as the \textbf{vertex} of the angle, and $OA$ and $OB$ are the
\textbf{sides or arms} of the angle.

\begin{tikzpicture}
  \draw
     (3,-1) coordinate (a) node[right] {A}
    -- (0,0) coordinate (o) node[left] {O}
    -- (2,2) coordinate (b) node[above right] {B}
    pic["$\alpha$", draw=orange, <->, angle eccentricity=1.2, angle radius=1cm]
    {angle=a--o--b};
\end{tikzpicture}

The angle is called angle $AOB$ or angle $BOA$ and is written as $\angle AOB$ or
$\angle BOA$. Another way of writing this angle is $A\hat O B$ or $B\hat O A$.
When it is clear which angle we are referring to, we may also call it angle $O$
and write it as $\angle O$ or $\hat O$

Types of Angles

\begin{tabular}{|l|l|l|}
  \hline
  Name &  Definition & Illustration \\
  \hline
  Acute angle & $0^{\circ} < x^{\circ} < 90^{\circ}$ &
\begin{tikzpicture}
  \draw
     (3,-1) coordinate (a) node[right] {}
    -- (0,0) coordinate (o) node[left] {}
    -- (2,2) coordinate (b) node[above right] {}
    pic["$\alpha$", draw=orange, <->, angle eccentricity=1.2, angle radius=1cm]
    {angle=a--o--b};
\end{tikzpicture}
  \\
  \hline
\end{tabular}


\end{document}