\documentclass[../main]{subfiles}

\pgfplotsset{every axis/.append style={
                    axis x line=middle,    % put the x axis in the middle
                    axis y line=middle,    % put the y axis in the middle
                    axis line style={<->}, % arrows on the axis
                    xlabel={$x$},          % default put x on x-axis
                    ylabel={$y$},          % default put y on y-axis
                    }}
\tikzset{>=stealth}

\begin{document}

\section{Primes, Highest Common Factor and Lowest Common Multiple}

\subsection{Primes}

\subsubsection{Factors and Multiples}
\textbf{Factors} of a number are whole numbers that multiply to give that number.

Notice: 1 is not a prime number.

Whole numbers: 0,1,2,3,$\cdots$

\textbf{Even numbers} are whole numbers that are divisible by 2, e.g. 0, 2, 4,
6, 8, $\cdots$

\textbf{Odd numbers} are whole numbers that are not divisible by 2, e.g. 1, 3,
5, $\cdots$


A \textbf{multiple} of a number is obtained when the number is multiplied by a
non-zero number.

\subsubsection{Prime Numbers,Prime Factorisation and Index Notation}    

A \textbf{prime number} is a whole number greater than 1 that has only two
factors, 1 and itself.


A \textbf{composite number} is a whole number greater than 1 that has more
than two factors.

A prime number which is a factor of a composite number is called the
\textbf{prime factor} of the composite number.

\textbf{Sieve of Eratoshenes}
\(5^4\) ( read as '5 to the \textbf{power} of 4') where 4 is called the
\textbf{index}(plural: indices). \(5^4\) is called the \textbf{index notation}
of \(5\times 5 \times 5\).

The \textbf{Fundamental Theorem of Arithmetic} states that 'every whole number
greater than 1 is either a prime number or it can be expressed as a
\textit{unique product} of its prime factors, where 'unique product' means that
there is only one product(where the order of the prime factors does not matter).

A composite number can be expressed as a product of prime factors only by
\textbf{prime factorisation}. There are two methods of prime factorisation:

\begin{enumerate}
\item Factor tree
\item Repeated division
\end{enumerate}

Worked Example:

Find the prime factorisation of 486.

\textbf{Worked Solution:}

\underline{Method 1 factor tree}

\underline{Mehtod 2(repeated devision)}


Prime factorisation of $486= 2\times 3 \times 3 \times 3 \times 3 \times 3
\times 1 = 2 \times 3^5$ 


\subsubsection{Squares,Square Roots,Cubes and Cube Roots}
A \textbf{square} of a number is the product when the number is multiplied by
itself.

The factor of a square is known as a \textbf{square root}.

Notice: \begin{itemize}
  \item Numbers whose square roots are whole numbers are known as
\textbf{perfect squares}.
\item $\sqrt{1}=1$
\item $\sqrt{x^2}=x$
\end{itemize}

Attention: For a number to be perfect square, the index of each prime factor
must be even. 

A \textbf{cube} of a number is the product when the number is multiplied by
itself thrice.

The factor of a cube is known as \textbf{cube root}.

Notice: The symbol $\sqrt[3]{}$ is used to denote cube root.
Numbers whose cube roots are whole numbers are known as \textbf{perfect cubes}.


Attention: For a number to be perfect cube, the index of each prime factor must
be a multiple of 3.

\textbf{Trial division}\textbf{Trial division } 
\subsubsection{Highest Common Factor(HCF)}
\textbf{Common factors} of two numbers are the factors that the two numbers have
in common.

The largest common factor of two or more numbers is called the \textbf{highest
  common factor(HCF)} of the numbers. It can be obtained by: 
\begin{enumerate}
\item prime factorisation
  \item repeated division
\end{enumerate}

\subsubsection{Lowest Common Multiple(LCM)}
A \textbf{common multiple} of two numbers is a number that is a multiple of both
numbers.

The smallest common multiple of two or more numbers is called the \textbf{lowest
  common multiple(LCM)} of the numbers. It can be obtained by prime
factorisation and repeated division.


\end{document}