\documentclass[../main]{subfiles}

\begin{document}

\section{Approximation and Estimation}

\subsection{Approximation}

To round off a number to a specified place, look at the digit on the right of
the specified place.

\begin{itemize}
\item If the digit is \textbf{less than 5}, replace the digit and all the digits
  to its right with zeros.
\item If the digit is \textbf{5 or greater than 5}, add 1 to the digit in the
  specified place and replace all the digits to its right with zeroes.
  
\end{itemize}

To round off a decimal to a specified decimal place, look at the digit on the
right of the specified decimal place.
\begin{itemize}
\item If the digit is \textbf{less than 5}, drop the digit and all the digits to
  its right (if any).
\item If the digit is \textbf{5 or greater than 5}, add 1 to the digit in the
  specified decimal place and drop all the digits to its right.
\end{itemize}

\subsection{Significant Figures}
The number of digits used to denote an exact value to a specified degree of
accuracy is called \textbf{significant figures}.

Rules for rounding a number to a given number of significant figures:

\begin{center}
  \begin{tabular}{|p{4cm} | p{4cm} |}
    \hline
    Rule    &   Examples \\
    \hline
    All non-zero digits are significant  &
                                           \begin{itemize}
                                           \item 9 \(\rightarrow\) 1 sig. fig.
                                           \item 52 \(\rightarrow\) 2 sig. fig.
                                           \item 158 \(\rightarrow\) 3 sig. fig.
                                           \item 6487 \(\rightarrow\) 4 sig. fig.
                                           \end{itemize} \\ \hline
    Zeros between non-zero digits are significant &
                                                    \begin{itemize}
                                                    \item 503 \(\rightarrow\) 3 sig. fig.
                                                    \item 8019 \(\rightarrow\) 4 sig. fig.
                                                    \item 25 007 \(\rightarrow\) 5 sig. fig.
                                                    \end{itemize} \\ \hline
    Zeros preceding the first non-zero digit are not significant &
                                                                   \begin{itemize}
                                                                   \item 0.001 \(\rightarrow\) 1 sig. fig.
                                                                   \item 0.389 \(\rightarrow\) 3 sig. fig.
                                                                   \end{itemize} \\ \hline
    Zeros following a non-zero digit after the decimal point are significant &
                                                                               \begin{itemize}
                                                                               \item  0.090 \(\rightarrow\) 2 sig. fig.
                                                                               \item  0.200 \(\rightarrow\) 3 sig. fig.
                                                                               \end{itemize} \\ \hline
    Zeros following a non-zero digit in a whole number may or may not be significant, depending on how the estimation is made.
    &
      \begin{itemize}
      \item  23 100 \(\rightarrow\) 3 sig. fig. if 23 099 is rounded off to 3 sig. fig.
      \item  23 100 \(\rightarrow\) 4 sig. fig. if 23 103 is rounded off to 4 sig. fig.
      \end{itemize} \\ \hline

  \end{tabular}
\end{center}

Common mistake:
\begin{itemize}
\item Many students 28 070 has 5 significant figures. The zero following the number 7 may or
  may not be significant, depending on how the estimation is made. Therefore, we
  cannot say that 28 070 has 5 significant figures.
\item It is incorrect to say that 100 000.20 has only 7 significant figures. We
  know that the zeros between non-zero digits are significant but do not forget
  that after the decimal point, the zero following a non-zero digit is also
  significant. Therefore, 100 000.20 has 8 significant figures.
 
\item We cannot assume that 70 000 with only 1 significant figure is incorrect.
  The zeros following the number 7 may or may not be significant, depending on
  how the estimation is made. Therefore, we cannot say that 70 000 does not have
  1 significant figure. 
\end{itemize}


\end{document}
