\documentclass[../main]{subfiles}

\begin{document}

\section{Linear Equations}

\subsection{Simple linear equations in one unknown}
An \textbf{equation} is a mathematical statement that contains two expressions
joined by an equal sign.

A \textbf{linear equation} is an equation of the form \(ax + b=c\), where \(a, b
and c\) are constants and \(a \neq 0\).

Note: the highest power of the variable in a linear equation is 1.


The value of the unknown variable that will make an equation true is called a
\textbf{solution} or \textbf{root} of the equation.

Rules for solving linear equation:
\begin{enumerate}
\item  A similar number can be added to or subtracted from both sides of the
  equation.

  Given the equation \(x =y\), a number \(k\) can be 
  \begin{itemize}
  \item  added to both sides of the equation such that \(x + k = y + k\)
  \item subtracted from both side of the equation such that \(x - k = y -k\)
  \end{itemize}

 
\item Each side of the equation can be multiplied or divided by a similar number
  except zero.
  
  Given the equation \(x =y\), a number \(k(k \neg 0)\) can be 
  \begin{itemize}
  \item multiplied by a number, such that \(kx = ky\),
  \item  divided by a number, such that \({\frac x k }= {\frac y k}\)
  \end{itemize}

 
\item  Rearrange the terms such that those containing variables are on one side
  and constant terms are on the other side.
  
\end{enumerate}

Worked example:
\(x + 12 = 50\)

\begin{align*} 
  x + 12 &=  50 \\
  x + 12 -12 &=  50 -12 \\
  x &= 38
\end{align*}

Note: Instead of adding, subtracting, multiplying and dividing both the LHS and
RHS with the some terms, the equation can be solved by bringing over terms from
on side to the other. For example.

\begin{align*}
  \frac {9y + {\frac 2 3}} 4 &= 54 \\
  9y + {\frac 2 3} &= 4(54)  \\
  9y &= 216 - {\frac 2 3 }    \\
  y &= {\frac {\frac {646} 3} 9}  \\
  y &= 23 { \frac {25} {27}}
\end{align*}
\subsection{Equations involving brackets}
When the given equation involves brackets, the brackets need to be removed
first.

The distributive law of multiplication over addition or subtraction can be
applied when solving equations involving brackets.

Note:
The general distributive law of multiplication over addition:
\(a(x +y ) = ax + ay\) 


The general distributive law of multiplication over subtraction:
\(a(x -y ) = ax - ay\) 

\subsection{Simple Fractional Equations}
A \textbf{fractional equation} is an equation that has an unknown in the
denominator of its term.


To solve a fractional equation, the equation needs to be transformed into a
linear equation by multiplication.

\begin{align*}
  {\frac{12}{x -9}} &= 3 \\
  (x-9)(\frac {12} {x -9}) &= 3 (x 09) \\
  12 &= 3x -27 \\
  12 + 27 & = 3x \\
  3x &= 39 \\
  x &= {\frac {39} 3} \\
  x &= \mathbf{13}
\end{align*}

Note: It is important to check whether the solution obtained satisfies the given
fractional equation.

\subsection{Formulae}

Linear equations can be used to solve real-life problems.

Steps involved in problem-solving with linear equations
\begin{enumerate}
\item Read the question and identify the unknown quantity.
\item Represent the unknown quantity using a letter, eg. a, b, x, or y.
\item Express other quantities in terms of the unknown quantities, eg. a, b, x,
  or y.
\item Form an equation based on the given information.
\item Solve the equation.
\item Write down the answer statement.
  
\end{enumerate}

\end{document}