\documentclass[../main]{subfiles}

\begin{document}

\section{Integers, Rational Numbers and Real Numbers}

\subsection{Negative Numbers, Positive Numbers and Integers}
\textbf{Negative numbers} are numbers with negative sign(-). For example, -1,
-2, -3.6 and - \(\frac 1 2 \) are negative numbers.

\textbf{Positive numbers} are numbers with positive sign(+) or without any sign.

\textbf{Integers} are the numbers ...,-3,-2,-1,0,1,2,3,...
\begin{enumerate}
  \item Integers that are greater than zero are known as \textbf{positive
    integers}
  \item Integers that are less than zero are known as \textbf{negative integers}
\end{enumerate}
Notice: Integers comprises of the number zero, 0 and all positive numbers and
negative numbers that are whole numbers.

Common mistake:
It is wrong to say that all positive and negative numbers are integers. Not all
positive and negative numbers are whole numbers. However, it is correct to say
that all integers are whole numbers.

A number line can be used to show the order of numbers. On a horizontal number
line,
\begin{enumerate}
\item all positive numbers are positioned to the right of zero.
\item all negative numbers are positioned to the left of zero,
\item numbers are arranged in ascending order from left to right,
  \item every number is greater than the number(s) on its left and less than the
    number(s) on its right.
    
\end{enumerate}

The symbol \textless , \textgreater ,  \(\leq\), and \(\geq\) are called \textbf{inequality signs}.
\begin{center}
\begin{tabular}{| c| l |  }
  \hline
 Inequality Signs & Meaning \\ 
  \hline
  \textless & Less than \\  
  \hline
 \textgreater  & Greater than \\   
  \hline

 \(\leq \) & Less than and equal to \\   
  \hline
 \(\geq \) & Greater than and equal to \\   
  \hline
\end{tabular}
\end{center}

\subsection{Addition and Subtraction of Integers}

Addition of a \textbf{positive number} refers to a movement to the \textbf{right} of a
number line.

Addition of a \textbf{negative number} refers to a movement to the \textbf{left}
of a number line.
 
Subtraction of a \textbf{positive number} refers to movement to the
\textbf{left} of a number line.

Subtraction of a \textbf{negative number} refers to a movement to the
\textbf{right} of a number line.

Notice: Given that a \textgreater b,
\begin{itemize}
\item \(a + (-b) = a - b \)
\item \((-a) +b = -(a - b)\)
\item \((-b)+ a = a -b\)
\item \((-a)+(-b)= -(a+b)\)
\end{itemize}

\subsection{Multiplication, Division, and Combined Operations of Integers}
Rules of multiplying integers:
\begin{enumerate}
\item \((+a) \times (+b) = +(a \times b)\)  
\item \((-a) \times (-b) = +(a \times b)\) 
\item \((+a) \times (-b) = -(a \times b)\)
\item \((-a) \times (+b) = -(a \times b) \)

\end{enumerate}

\end{document}