\documentclass[../main]{subfiles}

\begin{document}

\section{Number Sequances}
\subsection{Number Sequances}
A \textbf{number sequence} is an ordered list of numbers generated according to
a particular rule.

Each number in a sequence is called a \textbf{term}.

\subsection{General Term of a number sequence}
The \textbf{general term} of a sequence is the $n$th term of the sequence.

Any term in a given sequence can be determined by substituting the corresponding
value for $n$ in the formula for the general term.

The general term of a sequence is given as \(T_n=(\frac {3n-1} n)^2\), Find
\begin{enumerate}[label=(\alph{*})]
\item the 9th term,
 
\item the difference between the 100th term and 50th term 
\end{enumerate}

Worked Solution:
\begin{enumerate}[label=(\alph{*})]
\item substitute $n$=9 into \(T_n={\frac {3n-1} n}^2\).
\end{enumerate}

\subsection{Number Patterns}
Steps in solving probems involving number patterns:
\begin{enumerate}[label=(\alph{*})]
\item Understand the problem.
\item Look for a pattern. Draw diagrams if necessary.  
\item Generalise the results obtained.
\item Check the validity of the solution.
\end{enumerate}

\section{Percentage}
\subsection{Introduction to percentage}
A percentage is a fraction where the denominator is 100. '\%' is the symbol used
to represent per cent. Per cent means per hundred.

To convert a fraction or a decimal into a percentage, we multiply the fraction
or decimal by 100\%.

To convert a percentage into a fraction or decimal, we divided the fraction or
decimal by 100\%.

\subsection{Percentage change}

To express one quantity ($x$) as a percentage of another quantity ($y$), change
the expression into a fraction and multiply by 100.

A percentage can be used to compare two or more quantities.

\subsection{Reverse percentage}
\subsection{Percentage in practice situations}

\subsection{Discount and Commission}
A discount is calculated as the difference between the market price and the sale
price.

Market price is also known as the original selling price and the sale price is
the existing selling price.

A commission is a profit earned by an agent when s/he sells something on behalf
of another person. It is usually expressed as a percentage of the cost price or
the selling price.

\subsection{Value-added tax and GST}
The Singapore government added a 7\% tax to the cost of goods and services. This
is known as the goods and services tax(GST). In other countries, such tax is
known as value-added tax(VAT).


\subsection{Other Financial Transactions}
\subsubsection{Profit and Loss}

Profit(Increase) = Selling Price - Cost price

\[Percentage\ profit={\frac {profit} {Cost\ price}} \times 100\%\]

Loss(Decrease) = Cost price - Selling price

\[  Percentage loss = {\frac {loss} {cost\ price}} \times 100\%\]

\subsubsection{Simple Interest and compound interest}

Simple interest(I) = \({\frac {PRT} {100}}\)

I = simple interest
P = principal(Amount borrowed/invested)
R = interest rate( per year)
T = Time(years)

\subsubsection{Hire purchase}

\section{Ratio, Rate and Speed}
\subsection{Ratio}
A \textbf{ratio} is a comparison of two \textbf{similar quantities} and it has
no unit.

The ratio of $x$ to $y$ is defined as follows:
\(x:y={\frac x y}, where\ y \neg 0\)

Note: \(x:y \neg y:x\)

Since a ratio can be expressed as a fraction, it is important to express a ratio
in its lowest term.

Given the ratio \(x:y\) and any integer $m$ where \(m \neg 0\), \(x:y = mx:my =
{\frac x m}:{\frac y m}\)

\(mx:my\) and \({\frac x m}:{\frac y m}\) are the \textbf{equivalent ratios} of
\(x:y\), where $x$ and $y$ can by any rational number and \(y \neg 0\).

A ratio can also be used to represent three quantities. HOW?

To find \(x:y:z\) given \(x:y\) and \(y:z\),:
\begin{enumerate}
\item If the values of $y$ in the given ratio is the same, we can write them as
  a single ratio \(x:y:z\).
 
\item If the values of $y$ in the given ratios are different, each ratio has to
  be converted to an equivalent ratio where the new value of $y$ is the LCM of
  the original values of $y$ .
  
\end{enumerate}

Given that \(a:b=1:4\) and \(b:c=12:13\), find \(a:b:c\)

\(a:b=1:4 = 3:12\)
\(a:b:c=3:12:13\)

\subsection{Rate}
\textbf{Rate} involves two different quantities. It is calculated to show the
change in a quantity per unit of another quantity.

Sometimes, a rate may not be a constant throughout a given unit. In such case,
the rate calculated is known as the \textbf{average rate}.

\subsection{Speed}
\textbf{Speed} is the rate of distance travelled per unit time.
\[Speed = {\frac {Distance\ travelled} {Time\ taken}}\]


Speed that is unchanged during a period of time is known as a \textbf{uniform
  speed} or a \textbf{constant speed}.

When the speed of an object is not uniform throughout a period of time,
\textbf{average speed} is calculated.
\[Average\ speed={\frac {Total\ distance\ travelled} {Total \ time \ taken}}\]



\end{document}