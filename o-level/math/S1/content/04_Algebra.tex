\documentclass[../main]{subfiles}

\pgfplotsset{every axis/.append style={
                    axis x line=middle,    % put the x axis in the middle
                    axis y line=middle,    % put the y axis in the middle
                    axis line style={<->}, % arrows on the axis
                    xlabel={$Re(z)$},          % default put x on x-axis
                    ylabel={$Im(z)$},          % default put y on y-axis
                    }}
\tikzset{>=stealth}

\begin{document}

\section{Basic Algebra}

\subsection{Fundamental Algebra}
In algebra, letters are used to represent numbers.

A \textbf{variable} is a letter that is used to represent an unknown number or
some unknown number(s).

algebraic notations:

\begin{center}
  \begin{tabular}{| p{4cm} | p{4cm} |}
    \hline
    
    \(a + b = b + a =c \) & The sum of $a$ and $b$ (or $b$ and $a$) is $c$. \\
    \hline
    \(a-b=c\) & The difference between $a$ and $b$ is $c$, where $a$ is more than $b$ \\
    \hline
    for \(a\times b = b \times a = c\), we usually write \(ab=c or ba=c\) & The producti of $a$ and $b$ (or $b$ and $a$) is $c$. \\
    \hline
    \(a \times a = a^2\) & The square of $a$ is $a^2$ \\
    \hline
    \(a \times a \times a= a^3\) & The cube of $a$ is $a^3$ \\
    \hline
    For \(a \div b = a \times {\frac 1 b} = c\), we usually write \({\frac a b}=c\) & The quotient of $a$ when divided by $b$ is $c$, where
                                                                                      \(b \neg 0\) \\
    \hline                                                                                      
    
  \end{tabular}
\end{center}

An \textbf{algebraic expression} involves numbers and variables that are
connected with operational signs(such as \(+, -, \times and \div\)).

\(5x\) is \textbf{not an algebraic expression}, \(5x\) is a term.

\(8p - 2q\) is \textbf{an algebraic expression}.

\(2a +7b = 105\) is an algebraic equation.

The terms in algebra can be expressed in \textbf{index notation}. For example,
\(p \times p \times p\) can be expressed as \(p^3\).

Common mistake:
\(36m + 6n = 42mn \) is wrong! Terms of different variables cannot be added.
\(12m \times 3 + 54mn \div 9m = \frac {36m + 54mn} 9m = 4 + 6n\) is wrong!
Always perform multiplication and division before addition and subtraction.

\subsection{Evaluation of algebraic expressions and formulae}
To evaluate an algebraic expression, substitute each variable with the given
value.

A \textbf{formula} is general mathematical statement or rule.
Given that p = -10, q =2, and r = 6, evaluate 
\(8p - 19q +25r\)
\begin{align*}
 8p-19q + 25 r &= 8(-10) - 19(2)+ 25(6)\\
               &= -80-38 +150 \\
               &=\mathbf{32}
\end{align*}

\section{Algebraic Manipulation}
\subsection{Like Terms and Unlike Terms}

An \textbf{algebraic expression} consists of \textbf{terms} connected by
operational signs \(+, -, \times or \div\). 

A term may consist of a number and a viable, a number only, or a viable only.
\begin{enumerate}
\item  The numerical part of the term is called the \textbf{coefficient} of the
  variable. 
\item  A term with no variable is called a \textbf{constant term}.
  
\end{enumerate}

Common mistake:
In the expression \(-3ab - 7a^2 + 2b^2 + 5a - 4b - 6\), it is wrong to say that
2b is the coefficient of b since b and \(b^2\) are two different variables.
In this case, 2 is the coefficient of \(b^2\).

In the expression, \(2p - 18q\), it is wrong to say that 18 is the coefficient
of \(q\),The negative sign has to be included, hence the coefficient of \(q\) is
-18.

If the given two terms have identical variables, they are known as \textbf{like
  terms}. If the variables are not identical, they are  \textbf{unlike terms}.

To simplify an algebraic expression,
\begin{enumerate}
\item  collect all the like terms together,
 
\item  add or subtract only the like terms. 
\end{enumerate}

\subsection{Addition and subtraction of linear algebraic expressions}
The rule is to \textbf{add or subtract like terms only}.

To perform addition or subtraction of the given expressions:

\begin{enumerate}
\item Remove the brackets. Take note of the change in signs when brackets are
  removed.
 
\item  Collect the like terms together,
 
\item Add or subtract the like terms.
\end{enumerate}

\subsection{Simplification of linear algebraic expressions}
The general distribution law of multiplication over addition:
\(a(x + y)= ax + ay\)

\(ax + ay \) is the expanded form of \(a(x + y)\)

The general distribution law of multiplication over subtraction:

\(a(x - y)= ax - ay\)

\(ax - ay \) is the expanded form of \(a(x - y)\)

If the given algebraic expression has  more than one  pair of brackets, simplify
the innermost pair of brackets first.

\subsection{Simplification of linear expressions with fractional coefficients}

To simplify algebraic fractions, express all fractions with a common denominator
which is the LCM of all denominators. 

\subsection{Factorisation by using common factors}
\textbf{Factorisation} is a process of writing an algebraic expression as a
product of its factors.

Given the expression \(ax+ay\),
\(ax + ay = a(x + y)\)

where \(a\) is the common factor of the terms \(ax\) and \(ay\), Here, the
factors of \(ax + ay\) are \(a\) and \(x + y\).

Note: Factorisation is the reverse of expansion.

Steps to factorise a given algebraic expression:
\begin{enumerate}
\item  Identify the common factor of the terms.
 
\item  Write each term as a product of the common factor and its remaining
  factors.
 
\item  Factorise the expression.
  
\end{enumerate}

\subsection{Factorisation by grouping terms}

Sometimes, we can arrange and divide the terms in the given expression into
groups such that the terms in each group have a common factor.

Steps to perform factorisation by grouping terms:
\begin{enumerate}
\item Array the terms into two groups.
 
\item  Identify the common factor of each group.
\item  Factorise each group
\item Factor out the common factor for each group.
  
\end{enumerate}


\end{document}