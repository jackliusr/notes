\documentclass[../main]{subfiles}

\begin{document}

\section{Simple Inequalities}

\subsection{Solving Simple Inequalities}
An \textbf{inequality} is similar to an equation except that an
\textbf{inequality sign}(\(<, >, \leqslant, or \geqslant\)) is in the place of
an equal sign.

To solve an inequality of the form \(ax < b\) where \(a > 0\), the following can
be applied:
\begin{enumerate}
\item If \(a < b\ and\ c > 0\) , then \(ac < bc\).
\item If \(a > b\ and\ c >0 \), then \(ac > bc\).
  
\end{enumerate}
Note: The above property is also true when '\(<\)' or '\(>\)' is replaced by
'\(\leqslant\)' or \(\geqslant\)' respectively. 

\subsection{Applications of Simple Inequalities}
Steps for solving problems involving inequalities:
\begin{enumerate}
\item Read the question and identify the unknown quantity.
\item Represent the unknown quantity by a letter.
 
\item Write an inequality, in terms of \(x\) , to represent the information
  given.
 
\item Solve the inequality.
\item State the solution to the problem.  
\end{enumerate}
\end{document}