\documentclass[../main]{subfiles}

\begin{document}

\section{Integers, Rational Numbers and Real Numbers}

\subsection{Negative Numbers, Positive Numbers and Integers}
\textbf{Negative numbers} are numbers with negative sign(-). For example, -1,
-2, -3.6 and - \(\frac 1 2 \) are negative numbers.

\textbf{Positive numbers} are numbers with positive sign(+) or without any sign.

\textbf{Integers} are the numbers ...,-3,-2,-1,0,1,2,3,...
\begin{enumerate}
\item Integers that are greater than zero are known as \textbf{positive
    integers}
\item Integers that are less than zero are known as \textbf{negative integers}
\end{enumerate}

Note: Zero, 0 , is neither positive nor negative.

Note: Integers comprises of the number zero, 0 and all positive numbers and
negative numbers that are whole numbers.

Common mistake:
It is wrong to say that all positive and negative numbers are integers. Not all
positive and negative numbers are whole numbers. However, it is correct to say
that all integers are whole numbers.

A number line can be used to show the order of numbers. On a horizontal number
line,
\begin{enumerate}
\item all positive numbers are positioned to the right of zero.
\item all negative numbers are positioned to the left of zero,
\item numbers are arranged in ascending order from left to right,
\item every number is greater than the number(s) on its left and less than the
  number(s) on its right.
  
\end{enumerate}

The symbol \textless , \textgreater ,  \(\leq\), and \(\geq\) are called \textbf{inequality signs}.
\begin{center}
  \begin{tabular}{| c| l |  }
    \hline
    Inequality Signs & Meaning \\ 
    \hline
    \textless & Less than \\  
    \hline
    \textgreater  & Greater than \\   
    \hline

    \(\leq \) & Less than and equal to \\   
    \hline
    \(\geq \) & Greater than and equal to \\   
    \hline
  \end{tabular}
\end{center}

\subsection{Addition and Subtraction of Integers}

Addition of a \textbf{positive number} refers to a movement to the \textbf{right} of a
number line.

Addition of a \textbf{negative number} refers to a movement to the \textbf{left}
of a number line.

Subtraction of a \textbf{positive number} refers to movement to the
\textbf{left} of a number line.

Subtraction of a \textbf{negative number} refers to a movement to the
\textbf{right} of a number line.

Notice: Given that a \textgreater b,
\begin{itemize}
\item \(a + (-b) = a - b \)
\item \((-a) +b = -(a - b)\)
\item \((-b)+ a = a -b\)
\item \((-a)+(-b)= -(a+b)\)
\end{itemize}



Notice: Given that a \textgreater b,
\begin{enumerate}
\item \(b -a = -(a - b)\)
\item \(-a -b = -(a + b)\)
\item \(a - (-b) = a + b\)
\item \(-a -(-b) = - (a-b)\)
\end{enumerate}


\subsection{Multiplication, Division, and Combined Operations of Integers}
\subsubsection{Rules of multiplying integers:}
\begin{enumerate}
\item \((+a) \times (+b) = +(a \times b)\)  
\item \((-a) \times (-b) = +(a \times b)\) 
\item \((+a) \times (-b) = -(a \times b)\)
\item \((-a) \times (+b) = -(a \times b) \)

\end{enumerate}


\subsubsection{Rules of dividing integers}
\begin{enumerate}
\item \((+a) \div (+b) = + ( a \div b)\)
\item \((-a) \div (-b) = +(a \div b)\)
\item \((+a) \div (-b) = - (a \div b)\)
\item \((-a) \div (+b) = -(a \div b)\)
\end{enumerate}

\subsubsection{Steps for performing combined operations of integers}
\begin{enumerate}
\item If there is more than 1 part of brackets, always evaluate the expression
  in the innermost pair of brackets first.
\item After which, if there are any indices, evaluate them
\item Then perform multiplication and division from left to right.
\item Finally, perform addition and subtraction from left to right.
\end{enumerate}

Note: Evaluate the power of the numbers first before carrying out multiplication
and division operations.

\subsection{Rational Numbers}

A \textbf{rational number} is a number that can be expressed as a fraction
\(\frac a b\), where a and b are integers and b \(\neq\)=0.

Common mistake:
It is wrong to say that 0.125 is not rational number. In fact, 0.125 can be
expressed as \(\frac 125 100\), which is \(\frac 1 8\) when simplified. Hence
0.125 is a rational number.

Rules of adding or subtraction rational numbers:
\begin{enumerate}
\item Express the given rational numbers as equaivalent fractions with the
  \textbf{same denominator} before performing addition and subtraction.
\item Apply the rules of addition or subtraction of integers.
\end{enumerate}


Rules of multiplying or dividing two rational numbers, \(\frac a b \) and
\(\frac c d\):

\begin{enumerate}
\item \({\frac a b} \times {\frac c d} = {\frac {a \times c } {b \times d}}\),
  where \(\frac a b\) and \(\frac c d\) are 2 rational numbers, \(b \neq 0, c
  \neq 0\).
\item \({\frac a b} \div {\frac c d } = {\frac a b } \times {\frac d c} =
  {\frac {a \times d} {b \times c}}\)
\end{enumerate}

Each \textbf{positive} number, \(x\), has a \textbf{positive} square root
\(\sqrt x\) and a \textbf{negative} square root \(- \sqrt x\). For example,
16 has a square root of 4 and -4 because 16 = 4 x 4 and 16=(-4)x(-4).  


\begin{enumerate}
\item Each \textbf{positive} number has a \textbf{positive} cube root. For
  example, 27 has a cube root of 3 because 27 = 3 x 3 x 3.
\item Each \textbf{negative} number has a \textbf{negative} cube root. For
  example, -27 has a cube root of -3 because -27 = (-3) x (-3) x (-3). 
\end{enumerate}

\subsection{Real Numbers}
To express a rational number as a decimal, divide its numerator by its
denominator. A rational number can be expressed as a decimal by dividing its
numerator by its denominator.

\({\frac 7 8} = 7 \div 8 = 0.875 \)

Note: 0.875 is an example of a \textbf{terminating decimal}. Terminating
decimals have a finite number of digits.

\({\frac 8 15} = 8 \div 15 = 0.533 33\ldots = 0.5 \dot 3\)

Note: 0.533... is an example of a \textbf{repeating} or \textbf{recurring
  decimal}. A repeating decimal has one or more digits that are repeating
infinitely. A dot is placed above the repeating digit to show that it is
repeating.


Real numbers include
\begin{enumerate}
\item all rational numbers that are either terminating decimals or repeating or
  recurring decimals.
\item all non-terminating and non-repeating or recurring decimals. 
\end{enumerate}

\begin{forest}
  for tree={align=center, parent anchor=south, child anchor=north, l sep=5mm, rounded corners=2pt, fill=blue!20}
  [Real Numbers
  [Rational Numbers
  [Fractions]
  [Integers
  [Negative integers]
  [Whole numbers
  [Zero(0)]
  [Positive integers]]
  ]
  ]
  [Non-terminating and \\
  non-repeating or \\
  recurring decimals]] 
\end{forest}
\end{document}