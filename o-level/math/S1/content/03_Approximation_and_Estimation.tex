\documentclass[../main]{subfiles}

\begin{document}

\section{Approximation and Estimation}

\subsection{Approximation}

To round off a number to a specified place, look at the digit on the right of
the specified place.

\begin{itemize}
\item If the digit is \textbf{less than 5}, replace the digit and all the digits
  to its right with zeros.
\item If the digit is \textbf{5 or greater than 5}, add 1 to the digit in the
  specified place and replace all the digits to its right with zeroes.
  
\end{itemize}

To round off a decimal to a specified decimal place, look at the digit on the
right of the specified decimal place.
\begin{itemize}
\item If the digit is \textbf{less than 5}, drop the digit and all the digits to
  its right (if any).
\item If the digit is \textbf{5 or greater than 5}, add 1 to the digit in the
  specified decimal place and drop all the digits to its right.
\end{itemize}

\subsection{Significant Figures}
The number of digits used to denote an exact value to a specified degree of
accuracy is called \textbf{significant figures}.

Rules for rounding a number to a given number of significant figures:

\begin{center}
  \begin{tabularx}{\textwidth}{| l | l |}
    \hline
    Rule    &   Examples \\
    \hline
    All non-zero digits are significant  & ttt
  \end{tabularx}
\end{center}


\end{document}
