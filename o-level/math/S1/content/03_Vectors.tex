\documentclass[../main]{subfiles}

\newcommand*\colvec[3][]{
    \begin{pmatrix}\ifx\relax#1\relax\else#1\\\fi#2\\#3\end{pmatrix}
}

\begin{document}

\section{Vectors}

\subsection{Representation}

	\subsubsection{Point Representation}
	\(O\) is always defined as the origin \\
	Written: \underline{r} or \( \overrightarrow{OR} \) \\
	Column: \( \mathbf{r} = \colvec[a]{b}{c} \) \\
	Unit Vector: \( \mathbf{r} = a\mathbf{i} + b\mathbf{j} + c\mathbf{k} \) where \(\mathbf{i}\),\(\mathbf{j}\) and \(\mathbf{k}\) are unit vectors in the x, y and z dimensions
	\subsubsection{Line Representation}
	Vector Equation:
	\[ l: \mathbf{r} = \mathbf{a} + \lambda \mathbf{b} \quad \lambda \in \mathbb{R} \]
	Parametric Equation:
	\begin{equation*}
		l:\begin{cases}
			x = a_x + \lambda b_x \\
			y = a_y + \lambda b_y \quad \lambda \in \mathbb{R} \\
			z = a_z + \lambda b_z 
		\end{cases}
	\end{equation*}
	Cartesian Equation:
	\[ \frac{x-a_x}{b_x} = \frac{y-a_y}{b_y} = \frac{z-a_z}{b_z} \]
	\subsubsection{Plane Representation}
	Vector Equation:
	\[ \pi : \mathbf{r} = \mathbf{a} + \lambda \mathbf{b} + \mu \mathbf{c} \]
	Scalar Product Equation:
	\[ \pi : \mathbf{r} \cdot \mathbf{n} = d \]
	Cartesian Equation:
	\[ \pi: xn_x + yn_y + zn_z = d \]

\subsection{Manipulation}

	\subsubsection{Vector Algebra}
	Vector Addition: remove same inside or outside terms
	\[ \overrightarrow{OA} + \overrightarrow{AB} = \overrightarrow{OB} \]
	Negative Vectors: reverse the points
	\[ \overrightarrow{AB} = -\overrightarrow{BA} \]
	Vector Subtraction: reverse points, then add
	\[ \overrightarrow{OB} - \overrightarrow{OA} = \overrightarrow{OB} + \overrightarrow{AO} = \overrightarrow{AB} \]
	\subsubsection{Vector Properties}
	Modulus / Magnitude : The total distance of a vector
	\[ \bigg|\colvec[x]{y}{z}\bigg| = \sqrt{x^2+y^2+z^2} \]
	Unit Vector / Cap : A vector which defines a direction and has modulus of 1
	\[ \mathbf{\hat{a}} = \frac{1}{|\mathbf{a}|}\mathbf{a} \]
	\subsubsection{Ratio Theorem}
	Points between two direction vectors \(\mathbf{a}\) and \(\mathbf{b}\) are in the form:
	\[ \mathbf{r} = \frac{\mu\mathbf{a} + \lambda\mathbf{b}}{\mu + \lambda} \]
	\subsubsection{Scalar Product}
	Scalar/Dot product produces a scalar which is a representation of how inline two vectors are with each other.
	\[ \colvec[x_1]{y_1}{z_1} \cdot \colvec[x_2]{y_2}{z_2} = x_1x_2 + y_1y_2 + z_1z_2 \]
	\[ \mathbf{a} \cdot \mathbf{b} = | \mathbf{a}| | \mathbf{b}| cos(\theta) \]
	\[ \mathbf{a} \cdot \mathbf{a} = | \mathbf{a}| ^2 \]
	\[ \mathbf{a} \cdot (\perp \mathbf{a}) = 0 \]
	\[ \mathbf{a} \cdot (\mathbf{b \pm c}) = \mathbf{a} \cdot \mathbf{b} + \mathbf{a} \cdot \mathbf{c} \]
	\[ (\lambda \mathbf{a}) \cdot \mathbf{b} = \lambda (\mathbf{a} \cdot \mathbf{b}) \]
	\subsubsection{Vector Product}
	Vector/Cross product produces a vector which has direction perpendicular to its input vectors and has magnitude similar to area subtended by its input vectors.
	\[ \colvec[x_1]{y_1}{z_1} \times \colvec[x_2]{y_2}{z_2} = \colvec[y_1z_2-z_1y_2]{z_1x_2-x_1z_2}{x_1y_2-y_1x_2} \]
	\[ \mathbf{a} \times \mathbf{b} = \mathbf{\hat{n}}| \mathbf{a}| | \mathbf{b}| sin(\theta) \]
	\[ \mathbf{a} \times \mathbf{a} = 0 \]
	\[ \mathbf{a} \times (\perp \mathbf{a}) = \mathbf{\hat{n}}| \mathbf{a}| | \mathbf{b}|  \]
	\[ \mathbf{a} \times (\mathbf{b \pm c}) = \mathbf{a} \times \mathbf{b} + \mathbf{a} \times \mathbf{c} \]
	\[ (\lambda \mathbf{a}) \times \mathbf{b} = \lambda (\mathbf{a} \times \mathbf{b}) \]

\subsection{Angles Between Vectors}
	General formula:
	\[ cos(\theta) = \frac{\mathbf{a} \cdot \mathbf{b}}{|\mathbf{a}||\mathbf{b}|} \]
	Lesser used formula:
	\[ sin(\theta) = \frac{|\mathbf{a} \times \mathbf{b}|}{|\mathbf{a}||\mathbf{b}|} \]
	\subsubsection{Point-Point Angle}
	Let \(\mathbf{a}\) and \(\mathbf{b}\) be point vectors.
	\subsubsection{Line-Line Angle}
	Let \(\mathbf{a}\) and \(\mathbf{b}\) be direction vectors of lines.
	\subsubsection{Line-Plane Angle}
	Let \(\mathbf{a}\) be direction vector of line and \(\mathbf{b}\) be normal vector of plane \\
	Angle between line and plane will be \(90 \degree - \theta\).
	\subsubsection{Plane-Plane Angle}
	Let \(\mathbf{a}\) and \(\mathbf{b}\) be normal vectors of planes.

\subsection{Intersection}
	
	\subsubsection{Point-Line Intersection}
	Solve series of parametric equations or find \(\lambda\) which lets point equal to point on line.
	\subsubsection{Line-Line Intersection}
	If direction vectors are scalar multiples of each other, lines are parallel. \\
	Find a point which satisfies both lines, solving parametric equations of both line equations. \\
	\[ \mathbf{a_1} + \lambda \mathbf{b_1} = \mathbf{a_2} + \mu \mathbf{b_2} \quad \lambda, \mu \in \mathbb{R} \]
	If lines are both non-parallel and non-intersecting, lines are skew.
	\subsubsection{Line-Plane Intersection}
	If dot product of direction vector of line and normal of plane equals to \(0\), line is parallel to plane. \\
	Substitute line equation into plane equation and expand to solve for \(\lambda\).
	\begin{equation*} \begin{split}
		p & = (\mathbf{a} + \lambda \mathbf{b}) \cdot \mathbf{n}  \\
		  & = \mathbf{a} \cdot \mathbf{n} + \lambda (\mathbf{b} \cdot \mathbf{n})
	\end{split} \end{equation*}
	\subsubsection{Plane-Plane Intersection}
	If normal of planes are scalar multiples of each other, planes are parallel. \\
	Equating two planes results in a line. \\ 
	Cross product of normal vectors of two planes produces the direction vector of line. \\ 
	Position vector of line can be observed from equations, find a vector which satisfies both plane equations. \\ 

\subsection{Projections}

	\subsubsection{Point-Point Projection}
	To find distance \(d\) of projection of point vector \(\mathbf{a}\) on point vector \(\mathbf{b}\) :
	\[ d = \mathbf{a} \cdot \mathbf{\hat{b}} \]
	\subsubsection{Point-Line Projection}
	To find distance \(d\) of projection of point vector \(\mathbf{a}\) on direction vector of line \(\mathbf{b}\), similar to Point-Point Projection.
	\subsubsection{Point-Line Perpendicular}
	To find perpendicular of point vector \(\mathbf{a}\) on line \(l: \mathbf{r} = \mathbf{b} + \lambda \mathbf{c}\), let \(\mathbf{P}\) be a point on the line such that \(\overrightarrow{AP}\) is perpendicular to \(\mathbf{c}\) and solve for \(\lambda\) : 
	\begin{equation*}
		\begin{split}
		0 & = \overrightarrow{AP} \cdot \mathbf{c} \\
		  & = (\mathbf{b} + \lambda \mathbf{c} - \mathbf{a} ) \cdot \mathbf{c} \\
		  & = \lambda \mathbf{c} \cdot \mathbf{c} + (\mathbf{b} - \mathbf{a}) \cdot \mathbf{c}
		\end{split}
		\lambda | \mathbf{c} |^2 = (\mathbf{a} - \mathbf{b}) \cdot \mathbf{c}
	\end{equation*}
	To find distance \(d\) between point vector \(\mathbf{a}\) and its perpendicular on line with direction vector \(\mathbf{b}\), find magnitude of the cross product of \(\mathbf{a}\) and unit vector of \(\mathbf{b}\) :
	\[ d = |\mathbf{a} \times \mathbf{\hat{b}}| \]
	\subsubsection{Point-Plane Perpendicular}
	To find perpendicular of point vector \(\mathbf{a}\) on plane \( \pi : \mathbf{r} \cdot \mathbf{n} = d \), consider a line containing \(\mathbf{a}\) and with direction vector \(\mathbf{n}\), equate the two equations and then solve for \(\lambda\):
	\[ l = \mathbf{a} + \lambda \mathbf{n} \]
	\begin{equation*}
		\begin{split}
		d & = (\mathbf{a} + \lambda \mathbf{n}) \cdot \mathbf{n} \\
		  & = \mathbf{a} \cdot \mathbf{n} + \lambda \mathbf{n} \cdot \mathbf{n}
		\end{split}
		\lambda | \mathbf{n} |^2 = d - \mathbf{a} \cdot \mathbf{n}
	\end{equation*}
	To find distance \(d\) between point vector \(\mathbf{a}\) and plane with normal vector \(\mathbf{n}\), find projection of \(\mathbf{a}\) on unit vector of \(\mathbf{n}\):
	\[ d = |\mathbf{a} \cdot \mathbf{\hat{n}}| \]


\end{document}